\documentclass[12pt,a4paper,twoside,english]{memoir}
% % % % % % % % % % % % % % % % % % % % % % % % % % % % % % % % % % % % % % % % % % % % % % % % % % % % % % % % % %
% % % % % Ting og sager fra Esben % % % % % % % % % % % % % % % % % % % % % % % % % % % % % % % % % % % % % % % % %
% % % % % % % % % % % % % % % % % % % % % % % % % % % % % % % % % % % % % % % % % % % % % % % % % % % % % % % % % %


\usepackage{amsmath}
\usepackage{amssymb}
\usepackage[utf8]{inputenc} %utf8
\usepackage[english,danish]{babel}
\pdfoutput=1
\usepackage{graphicx}
\usepackage[T1]{fontenc}
%\graphicspath{{./figures/}}
\usepackage{url}

\usepackage{etoolbox}
\newtoggle{twoside}
\toggletrue{twoside}
\newtoggle{final}

\usepackage[bookmarks, pagebackref=false]{hyperref}
\usepackage[usenames,dvipsnames]{xcolor}
\definecolor{orange}{cmyk}{0,0.5,1,0}
\definecolor{rossoCP3}{cmyk}{0,.88,.77,.40}
\definecolor{moerkeroed}{cmyk}{0,.88,.77,.40}

\definecolor{oldhat}{rgb}{0.13671875, 0.15234375, 0.52734375}
\definecolor{nude}{rgb}{0.79296875, 0.5859375, 0.36328125}
\definecolor{wine}{rgb}{0.4296875, 0.0546875, 0.0546875}


\definecolor{graa}{rgb}{0.8,0.8,0.8}
\definecolor{moerkegraa}{cmyk}{0.67,0.58,0.54,0.09}
\definecolor{blaa}{rgb}{0.2,0.2,0.6}
		%
		\hypersetup{
			colorlinks, 
			bookmarksopen, 
			bookmarksnumbered,
			citecolor=oldhat, 		%color of links to bibliography
			linkcolor=rossoCP3,	%color of internal links
			urlcolor=rossoCP3,			%color of external links 
			}
\usepackage{bm}% bold math
\usepackage{bbm}
\usepackage{pxfonts}

\usepackage[section]{placeins}
\usepackage{xspace}
\usepackage{cancel} % to make the barred text notation

%
%
%Thesis packages
%
%
\usepackage[pass]{geometry}
\usepackage{youngtab}
%\usepackage{courier}

\usepackage{slashed}

%\usepackage[style=ieee,backend=bibtex,maxnames=10,maxcitenames=4,backref=false]{biblatex} %options for style: ieee, mla (terrible), nature, science (no article names)

\usepackage{longtable}			%Tables that span several pages

\newcommand{\ea}[1]{
\begin{align}
#1
\end{align}
}

\newcommand{\vectwo}[2]{
\left(
\begin{array}{c} #1 \\ #2 \end{array}
\right)
}

\newcommand{\HRule}{{\color{moerkegraa} \rule{\linewidth}{0.5mm}}} %  lines in the title page

\allowdisplaybreaks[1]

\setcounter{tocdepth}{2}


%\newcommand{\xxx}[1]{}
\newcommand{\xxx}[1]{{\color{rossoCP3} \textbf{#1}}\marginpar{$\bullet$}} %G�r en tekstreng r�d og fed, og laver en prik i margenen. God til kommentarer/ting der skal rettes.


\newcommand{\identity}{\mathbbm{1}}
\DeclareMathOperator{\tr}{Tr}
\newcommand{\MSbar}{$\overline{\textrm{MS}}$}
\renewcommand{\d}{\textrm{d}}

\usepackage{hyphenat}

\hyphenation{
  re-nor-ma-liz-ab-le
}


\def\CM{{\cal M}}
\newcommand{\mP}{{\bar M}_{P}}
\newcommand{\Pom}{I$\!$P}                % gives pomeron symbol
\def\lsim{\mathrel{\rlap{\lower4pt\hbox{\hskip1pt$\sim$}}
    \raise1pt\hbox{$<$}}}                % less than or approx. symbol
\def\gsim{\mathrel{\rlap{\lower4pt\hbox{\hskip1pt$\sim$}}
    \raise1pt\hbox{$>$}}}                % greater than or approx. symbol


\baselineskip=15pt

\setcounter{totalnumber}{10}
\renewcommand{\textfraction}{0.0}


\setcounter{secnumdepth}{2}

\let\footruleskip\undefined
\usepackage{fancyhdr}
\usepackage{pdfpages}
\newcommand\cleverpageing{\thepage}

\pagestyle{headings}
\pagestyle{fancy}
\fancyhead[LE,RO]{\slshape \thepage}
\fancyhead[LO]{\slshape \rightmark}
\fancyhead[RE]{\slshape \leftmark}
\fancyfoot[C]{}

\usepackage{lettrine}
\setlength{\DefaultNindent}{0em}
\setcounter{DefaultLines}{3}
%\newcommand{\fancyfont}{Elzevier}
\input{fonts/GoudyIn.fd}
\newcommand*\initfamily{\usefont{U}{GoudyIn}{xl}{n}}
\renewcommand{\LettrineFontHook}{\initfamily}
\newcommand{\initial}[2]{\lettrine{\color{moerkeroed}{#1}}{#2}}

% % % % % % % % % % % % % % % % % % % % % % % % % % % % % % % % % % % % % % % % % % % % % % % % % % % % % % % % % %
% % % % % Ting og sager fra migselv % % % % % % % % % % % % % % % % % % % % % % % % % % % % % % % % % % % % % % % %
% % % % % % % % % % % % % % % % % % % % % % % % % % % % % % % % % % % % % % % % % % % % % % % % % % % % % % % % % %

\usepackage{mathtools}
%
%% % % % % % % % % % % % % % % % % % % % % % % % % % % % % % % % % % % % % % % % % % % % % % % % %

\usepackage{booktabs}
%Giver følgende syntaks i tabular-environment: \toprule, \midrule, \bottomrule.
%\multicolumn{antal kolonner}{pos}{text} virker stadig.
%vandrette linier i dele af tabellen laves med \cmidrule(trimming){i-j}. i-j angiver kolonner, og 'trimming' erstattes med l, r eller lr.

\usepackage{float}
\newsubfloat{figure}
\usepackage{framed}
%Sætter en ramme om et område begrænset af \begin{framed} og \end{framed}.
\usepackage{ifthen}

\newcommand{\pp}[1]{\left( #1 \right)}
\newcommand{\sqb}[1]{\left\lbrack #1 \right\rbrack}

\newcommand{\citat}[2]{
	\ifthenelse{
		\equal{#2}{}%if no second argument
	}{%
	\cite{#1}\@\!\!%
}{%Else:
(\!\cite{#1}, #2)\@\!\!%
}
}
%\newcommand{\hest}[2]{\frac{\partial #1}{\partial #2}
\newcommand{\pardiff}[3][]{
	\ifthenelse{\equal{#1}{}}{
		\frac{\partial #2}{\partial #3}	
	}
	{
		\frac{\partial ^#1 \! #2}{\partial #3 ^#1}
	}
}
%%giver differentationsoperatoren (df/dx). F.eks. \pardiff[2]{f}{x}, eller \pardiff{}{x}
\newcommand{\diff}[3][]{
	\ifthenelse{\equal{#1}{}}{
		\frac{\ud #2}{\ud #3}	
	}
	{
		\frac{\ud ^#1 \! #2}{\ud #3 ^#1}
	}
}
\newlength{\testwd}
\newcommand{\fitpic}[1]{%
	\settowidth{\testwd}{\includegraphics{#1}}%
	\message{#1 width=\the\testwd, page=\the\textwidth}%
	\ifthenelse{\lengthtest{\testwd>\textwidth}}{%
		\noindent\includegraphics[width=\textwidth]{#1}}{%
		\centering\includegraphics{#1}\par}%
}
%Giver kommandoen \fitpic{fildestination}, som laver et billede så stort som muligt, dog ikke større end siden.
\DeclareMathOperator{\ud}{d\!}
% 'upright d'. Gør at du kan afslutte integraler med f.eks. \int x \ud x
\newcommand{\normdist}{\mathcal{N}(x;\mu,\sigma)}
\newcommand{\avg}[1]{\left\langle #1 \right\rangle}
\newcommand{\bracket}[2]{\langle #1 | #2 \rangle}
\newcommand{\ket}[1]{| #1 \rangle}
\newcommand{\bra}[1]{\langle #1 |}
\newcommand{\ssys}{S_{\text{system}}}
\newcommand{\cb}[1]{\left\{ #1 \right\}}

\definecolor{mygreen}{rgb}{0,0.6,0}
\definecolor{mygray}{rgb}{0.5,0.5,0.5}
\definecolor{mymauve}{rgb}{0.58,0,0.82}

% % %Stuff for python source
\usepackage{listingsutf8}
\lstdefinestyle{extfiles}{
	inputencoding=utf8/latin1,
	language=python,
	keywordstyle=\bfseries\ttfamily\color{moerkeroed}, %\color[rgb]{0,0,1},
	identifierstyle=\ttfamily,
	commentstyle=\color[rgb]{0.6, 0.6, 0.65},
	stringstyle=\ttfamily\color{oldhat}, %\color[rgb]{0.133,0.545,0.133},
	showstringspaces=false,
	basicstyle=\small,
	numberstyle=\footnotesize,
	numbers=left,
	stepnumber=1,
	numbersep=10pt,
	tabsize=2,
	breaklines=true,
	prebreak = \raisebox{0ex}[0ex][0ex]{\ensuremath{\hookleftarrow}},
	breakatwhitespace=false,
	aboveskip={1.5\baselineskip},
	columns=flexible,
%	extendedchars=true,
	%	backgroundcolor=\color{white},   % choose the background color; you must add \usepackage{color} or \usepackage{xcolor}
	basicstyle=\tiny,        % the size of the fonts that are used for the code
	breakatwhitespace=false,         % sets if automatic breaks should only happen at whitespace
	breaklines=true,                 % sets automatic line breaking
	captionpos=b,                    % sets the caption-position to bottom
	%	commentstyle=\color{mygreen},    % comment style
	%	deletekeywords={...},            % if you want to delete keywords from the given language
	%	escapeinside={\%*}{*)},          % if you want to add LaTeX within your code
	%	extendedchars=true,              % lets you use non-ASCII characters; for 8-bits encodings only, does not work with UTF-8
	frame=LRTB,                    % adds a frame around the code
	frameround = tttt,
	framerule = 0.4pt,
	rulecolor = \color{black},
	keepspaces=true,                 % keeps spaces in text, useful for keeping indentation of code (possibly needs columns=flexible)
	%	keywordstyle=\color{blue},       % keyword style
	%	morekeywords={*,...},            % if you want to add more keywords to the set
		numbers=left,                    % where to put the line-numbers; possible values are (none, left, right)
		numbersep=3mm,                   % how far the line-numbers are from the code
		numberstyle=\tiny\color{mygray}, % the style that is used for the line-numbers
		boxpos = c,
		framexleftmargin=0mm,
		framexrightmargin = -3mm,
	%	rulecolor=\color{black},         % if not set, the frame-color may be changed on line-breaks within not-black text (e.g. comments (green here))
	%	showspaces=false,                % show spaces everywhere adding particular underscores; it overrides 'showstringspaces'
	%	showstringspaces=false,          % underline spaces within strings only
	%	showtabs=false,                  % show tabs within strings adding particular underscores
	%	stepnumber=2,                    % the step between two line-numbers. If it's 1, each line will be numbered
	%	stringstyle=\color{mymauve},     % string literal style
	%	tabsize=2,                       % sets default tabsize to 2 spaces
	%	title=\lstname                   % show the filename of files included with \lstinputlisting; also try caption instead of title
		xrightmargin=-1.1cm,
		xleftmargin=-0.6cm
}


\lstdefinestyle{snippet}{
	inputencoding=utf8/latin1,
	language=python,
	keywordstyle=\bfseries\ttfamily\color{moerkeroed}, %\color[rgb]{0,0,1},
	identifierstyle=\ttfamily,
	commentstyle=\color[rgb]{0.6, 0.6, 0.65},
	stringstyle=\ttfamily\color{oldhat}, %\color[rgb]{0.133,0.545,0.133},
	showstringspaces=false,
	tabsize=4,
	breaklines=true,
%	prebreak = \raisebox{0ex}[0ex][0ex]{\ensuremath{\hookleftarrow}},
	breakatwhitespace=true,
%	aboveskip={1.5\baselineskip},
	columns=flexible,
	breaklines,
%	prebreak=\_
	basicstyle=\tiny,        % the size of the fonts that are used for the code
	breakatwhitespace=false,         % sets if automatic breaks should only happen at whitespace
	breaklines=true,                 % sets automatic line breaking
	captionpos=b,                    % sets the caption-position to bottom
	frameround = tttt,
	framerule = 0.4pt,
	rulecolor = \color{black},
	keepspaces=true,                 % keeps spaces in text, useful for keeping indentation of code (possibly needs columns=flexible)
	numbers=none,
	boxpos = c,
	framexleftmargin=0mm,
	framexrightmargin = 0mm,
	aboveskip=\baselineskip,
	belowskip=\baselineskip,
	xrightmargin=0cm,
	xleftmargin=0cm,
		frame=LRTB,                    % adds a frame around the code
		frameround = tttt,
		framerule = 0.4pt,
		rulecolor = \color{black}
}

\renewcommand{\lstlistingname}{Snippet}

\lstdefinelanguage{XML}
{
	morestring=[b]",
	morestring=[s]{>}{<},
	morecomment=[s]{<?}{?>},
	stringstyle=\color{black},
	identifierstyle=\color{moerkeroed},
	keywordstyle=\color{moerkeroed},
	morekeywords={xmlns,version,type}% list your attributes here
}

%For long pieces of source code.
\newcommand{\code}[1]{
	\iftoggle{final}{
		\raggedbottom
		\lstinputlisting[language=Python, style=extfiles]{#1}
	}{
		\includegraphics[width = 0.99\textwidth]{pics/code_placeholder.jpg}
	}
}

%For example files to be included as floats
\newcommand{\floatcode}[2][]{
%	\begin{minipage}{\linewidth}
		\lstinputlisting[float=htbp, style=snippet, language=python, #1]{#2}
%	\end{minipage}
}


% For very short snippets of code
\lstnewenvironment{snippet}[1][]%
{\hfill \mbox{} \noindent\minipage{\linewidth}
	\lstset{style=snippet, frame = none, #1}}
{\endminipage\hfill\mbox{}}


\usepackage[nice]{units}
% %\unit[val]{dim}
% %\unitfrac[val]{num}{den}

%\usepackage{SIunitx}

\newcommand{\fnorm}{\frac{1}{(2\pi)^{n/2}}}
\newcommand{\rint}{\int_{\mathbb{R}^n}}
\newcommand{\onenorm}{\frac{1}{\sqrt{2\pi}}}
\newcommand{\infint}{\int_{-\infty}^\infty}
\newcommand{\stdwidth}{0.9\textwidth}
\newcommand{\mb}[1]{\mathbf{#1}}

\newlength{\figwidth}
\setlength{\figwidth}{0.9\textwidth}

\newlength{\dualheight}
\setlength{\dualheight}{0.33\textheight}


\usepackage{todonotes}
\usepackage{lipsum}
\usepackage{microtype}

%Override the default part header page to mimick that of classicthesis
\renewcommand*{\beforepartskip}{\vspace*{0.21\textheight}}
\renewcommand*{\printpartname}{\partname}
\renewcommand*{\printpartnum}{\thepart}
\let\oldprintparttitle\printparttitle
\renewcommand*{\printparttitle}[1]{\normalfont\small\centering\color{moerkeroed}\textls[160]{\MakeTextUppercase{#1}}}

\newcommand{\e}{\mathrm{e}}

%Disable page numbers on part page
\makeatletter
\renewcommand\part{%
	\if@openright
	\cleardoublepage
	\else
	\clearpage
	\fi
	\thispagestyle{empty}%
	\if@twocolumn
	\onecolumn
	\@tempswatrue
	\else
	\@tempswafalse
	\fi
	\null\vfil
	\secdef\@part\@spart}
\makeatother

\newcommand{\bigo}[1]{\mathcal{O}\pp{#1}}